\documentclass{article}
\addtolength{\oddsidemargin}{-1.cm}
\addtolength{\textwidth}{2cm}
\addtolength{\topmargin}{-2cm}
\addtolength{\textheight}{3.5cm}
  \usepackage[utf8]{inputenc}
  \usepackage{titlesec}
  \usepackage[pdftex]{graphicx}
  \usepackage{natbib}
\usepackage{hyperref}
\hypersetup{
    colorlinks=true,
    linkcolor=blue,
    filecolor=magenta,      
    urlcolor=cyan,
}
  \setcounter{secnumdepth}{4}

  \newcommand\tab[1][1cm]{\hspace*{#1}}

  \begin{document}


 % generates the Cover page	
  \begin{titlepage}
	
	\begin{center}
		% Upper part of the page       
		\includegraphics[width=0.7\linewidth]{Images/uniLogo.jpg}\\
		% Title
		\rule{\linewidth}{0.5mm} \\[0.5cm]
		{ \huge \bfseries Cerebero}\\[0.3cm]
		\includegraphics[width=0.2\linewidth]{Images/cerebero.png}\\
		\rule{\linewidth}{0.6mm} \\[0.5cm] 		
  
		
		\begin{minipage}{0.4\textwidth}
			\begin{flushleft} \large
				Frederick Ehlers 
			\end{flushleft}
		\end{minipage}
		\begin{minipage}{0.4\textwidth}
			\begin{flushright} \large
				11061112
			\end{flushright}
		\end{minipage} \\[0.2cm]

		\begin{minipage}{0.4\textwidth}
			\begin{flushleft} \large
				 Jacobus Marais
			\end{flushleft}
		\end{minipage}
		\begin{minipage}{0.4\textwidth}
			\begin{flushright} \large
				15188397 
			\end{flushright}
		\end{minipage}\\[0.2cm]

		\begin{minipage}{0.4\textwidth}
			\begin{flushleft} \large
				Rikard Schouwstra
			\end{flushleft}
		\end{minipage}
		\begin{minipage}{0.4\textwidth}
			\begin{flushright} \large
				15012299
			\end{flushright}
		\end{minipage} \\[0.2cm]
		
		\begin{minipage}{0.4\textwidth}
			\begin{flushleft} \large
				Victor Twigge 
			\end{flushleft}
		\end{minipage}
		\begin{minipage}{0.4\textwidth}
			\begin{flushright} \large
				10376802
			\end{flushright}
		\end{minipage}\\[0.2cm]
		
		\begin{minipage}{0.4\textwidth}
			\begin{flushleft} \large
				Nardus van der Vyver 
			\end{flushleft}
		\end{minipage}
		\begin{minipage}{0.4\textwidth}
			\begin{flushright} \large
				15012698
			\end{flushright}
		\end{minipage} \\[0.2cm]
		
		\rule{\linewidth}{0.5mm} \\[0.5cm] 
		{ \huge \bfseries Stakeholders}\\[0.3cm]	
		\rule{\linewidth}{0.2mm} \\[0.5cm] 
		\begin{minipage}{0.4\textwidth}
			\begin{flushleft} \large
				\emph{} \\
				Computer Science Department of University of Pretoria:
			\end{flushleft}
		\end{minipage}
		\begin{minipage}{0.4\textwidth}
			\begin{flushright} \large
				\emph{} \\
				Vreda Pieterse
			\end{flushright}
		\end{minipage}\\[0.2cm]

		\rule{\linewidth}{0.2mm} \\[0.5cm] 
		\begin{minipage}{0.4\textwidth}
			\begin{flushleft} \large
				\emph{} \\
				eCivix
			\end{flushleft}
		\end{minipage}
		\begin{minipage}{0.4\textwidth}
			\begin{flushright} \large
				\emph{} \\
				Daniël Eloff {Chairperson}
			\end{flushright}
		\end{minipage}
		\rule{\linewidth}{0.2mm} \\[0.5cm] 
		
	\end{center}
\end{titlepage}
  

  \tableofcontents
  \newpage

\section{Introduction}
	We Cerebero are working together with eCivix to create a new web based game. The idea of the game is to create an election simulator to teach High school students how elections work and what their vote essentially means in the greater scheme. The user will create party that they will control. The game will revolve around the party gaining funds and man power to do campaigns and gain more funds and man power to run bigger and more effective campaigns. The user with the score at the end of the game wins. The user will be playing against an AI (Artificial Intelligence) player that we will program. The AI will try be more effective/ more successful than the user. There will be a leader-board with all the users' scores and at the end of the client's event a winner will be chosen for a prize. 
	


\section{Testing}
	\subsection{Mocha}
		\subsubsection{Why Mocha?}
			\begin{enumerate}
				\item Testing the AngularJS front end we will be making use of the testing framework called \href{http://mochajs.org/}{Mocha}. We choose Mocha as it runs on NodeJS and uses JavaScript to do the tests, which to our benefit because of the fact that AngularJS also works with JavaScript.
				\item This will then allow us to test our database Create, Read, Update and Delete operations that are carried out from the front end to ensure they are successfull and fail when they are supposed to.
				\item Mocha is strong but also deverse since we can utilise Mocha to do code coverage as well. Thus allowing us to see how well our tests cover all our code, only then we will truly know if we are testing for everything.
				\item There are numerous other reasons you want to use Mocha to improve our code:
					\begin{enurmate}
						\item It works with Continues Integration. Therefore our CI will not crash with Mocha tests.
						\item Supports node debugging 
						\item Support for extensible reporting 
						\item Much more benefits, feel free to see \href{http://mochajs.org/}{Mocha} for more benefits.
					\end{enurmate} 
			\end{enumerate}
           			
\section{Reporting}
   	\subsection{Reporting}
  
\section{Appendix}
	\subsection{Github}
		\href{https://github.com/KobusMarais/Cerebero}{Github}
	\subsection{Trello}
		\href{https://trello.com/b/WXh8cJZQ/demo-2}{Trello}
\end{document}
